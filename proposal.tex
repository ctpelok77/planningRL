\documentclass[10pt]{article}

\newcommand{\commentout}[1]{}
\usepackage[colorlinks=true,urlcolor=blue,citecolor=blue,pdfstartview=FitH]{hyperref}


\begin{document}

\title{Migrated to Overleaf. Please continue there.\\ \vspace*{0.7cm}ICAPS 2020 Workshop on \\ Bridging the Gap Between AI Planning and Reinforcement Learning (PRL)\\ \vspace*{0.7cm} Workshop Proposal
}
\date{}

\author{}

\maketitle

While AI Planning and Reinforcement Learning communities focus on similar
sequential decision-making problems, these communities remain somewhat unaware
of each other on specific problems, techniques, methodologies, and evaluation.
%
We propose to start a new workshop series at ICAPS 2020. This workshop aims to
bridge the gap between the AI Planning and Reinforcement Learning communities,
facilitate the discussion of differences and similarities in existing techniques,
and encourage collaboration across the fields. 
%
Possible topics of interest include
\begin{itemize}
\item How do concepts from each of these fields translate to the other?
%
\item Is there an interesting middle ground between capturing action dynamics
symbolically and model-free RL? Are there interesting domains that exhibit a
structure that can be partially captured with a symbolic model? How can that be
exploited in practice?
%
\item How to define what constitutes a solution and how to evaluate its quality
and solution guarantees? How to perform a meaningful comparison of different
methods?
%
\item How to evaluate and compare the computational efficiency of various
ML/symbolic/hybrid methods? Can we borrow/modify the methodology from the
learning track of the IPC?
%
\item What is the impact of whether the problem is repeatedly solved in
practice?
%
\item What are the settings that are both challenging and attractive for both
communities? 
%
One possible example: Factored-blackbox planning is a planning problem
formulated as a simulator where states are meaningful factors, and applying
actions lead to a new state. Some planning algorithms can achieve state of the
art on IPC benchmarks, where the action description has more information (PDDL).
Is that setting interesting for the RL community? How should methods be
evaluated?
\end{itemize}



\section*{Workshop Description}

The workshop seeks submissions that are relevant to both Reinforcement Learning
and Planning. 


%\newpage

\section*{Relevance to ICAPS 2020}
ICAPS 2020 has seen a significant increase in the number of submissions on
Reinforcement Learning, indicating an increased interest from the RL community
towards AI Planning.


\section*{Workshop Format}

The workshop is planned to have a full 1-day format, but the precise format is
to be decided as a function of the contributions received. We strive to have two
types of presentations: long and short (30 and 15 minutes, including
discussion), 1-2 invited talks, as well as 1-2 dedicated discussion sessions
where the audience members are encouraged to participate.

We would like to reduce the reviewing load of the community by not having a
program committee. Acceptance decisions will be made directly by the organizers.
We expect to have around 10-20 submissions, which can easily be handled by us,
particularly considering that reviewing for workshops is generally light.
Between us, we have sufficient competence in the topic of the workshop to make
informed decisions, but should the need arise, we can call on additional
reviewing expertise.


\section*{Diversity}


\section*{Organization}

% TODO: are we adding a PC?
% If so, let's split people.

\begin{itemize}

\item \href{http://web.engr.oregonstate.edu/~afern/}{Alan Fern}
  (\href{mailto:Alan.Fern@oregonstate.edu}{Alan.Fern@oregonstate.edu})\\
  
\item \href{https://www.upf.edu/web/vgomez}{Vicen\c{c} G\'{o}mez}
  (\href{mailto:vicen.gomez@upf.edu}{vicen.gomez@upf.edu})\\

\item \href{https://www.upf.edu/web/anders-jonsson}{Anders Jonsson}
  (\href{mailto:anders.jonsson@upf.edu}{anders.jonsson@upf.edu})\\
    
\item \href{https://resedit.watson.ibm.com/researcher/view.php?person=ibm-Michael.Katz1}{Michael Katz}
 (\href{mailto:michael.katz1@ibm.com}{michael.katz1@ibm.com})\\
 Michael is a researcher at IBM T.J. Watson Research Center, NY, USA. His
 primary research interest is in heuristic search for domain independent planning.
 He received his PhD for a dissertation on heuristics for domain independent
 planning from Technion in 2010.
 He was a co-organizer of the 2011, 2013, 2014, 2015, 2016, and 2018 H(S)DIP
 workshops.

\item \href{http://hectorpalacios.net/}{Hector Palacios}
  (\href{mailto:hectorpal@elementai.com}{hectorpal@elementai.com})\\
Hector is an Applied Research Scientist at Element AI, a start-up specialized in
creating AI-based products. In the industry, Hector has worked in NLP and
Vision. Lately,  he works on the use of AI reasoning methods in the context of
ML-based applications. In academia, Hector work in planning under incomplete
information. He holds a Ph.D. in Computer Science from Universitat Pompeu Fabra
(2009, Barcelona/Spain). His dissertation received the 2010 Best Dissertation
Award by ICAPS and an honourable Mention at the 2009 Artificial Intelligence
Dissertation Awards by ECCAI. Later on, he was awarded the IJCAI-JAIR Best Paper
Prize 2012 to an “outstanding paper published in JAIR in the preceding five
calendar years” alongside his Ph.D. advisor Hector Geffner.   

\item \href{http://d3m.mie.utoronto.ca}{Scott Sanner}
  (\href{mailto:ssanner@mie.utoronto.ca}{ssanner@mie.utoronto.ca})\\
Scott Sanner is an Assistant Professor in Industrial Engineering at the
University of Toronto and Cross-appointed in Computer Science.  Scott earned a
PhD in Computer Science from the University of Toronto (2008).  Scott’s research
spans a broad range of topics covering Machine Learning, Artificial
Intelligence, and Operations Research; he was a co-recipient of the 2014 AIJ
Prominent Paper Award and the 2016 Kikuchi-Karlaftis Best Paper Award of the
Transport Research Board.  Scott has served as Program Co-chair for the 26th
International Conference on Automated Planning and Scheduling (ICAPS 2016) and
previously organized workshops at AAAI (2016, 2017), ICAPS (2010, 2012, 2015),
NIPS (2011), and EWRL (2011).

% % Agreed to review
% % Does not want to be an organizer
% \item \href{https://sites.google.com/a/ualberta.ca/rickvalenzano/}{Rick Valenzano}
%   (\href{mailto:rick.valenzano@elementai.com}{rick.valenzano@elementai.com})\\


% \item \href{placeholder}{placeholder}
%   (\href{mailto:placeholder}{placeholder})\\

  
\end{itemize}

\section*{Estimated Interest}

The audience of this workshop is researchers that are interested in both
planning and reinforcement learning.
%
Based on our experience with new workshops, we expect to see a moderate
participation at first. We estimate an audience of about 15 participants.  

\subsection*{Possible Invited Speakers}

\begin{itemize}
  \item Hector Geffner
  \item Leslie Kaelbling
  \item Marc Bellemare
  \item Murray Shanahan
  \item Peter Stone
  \item George Konidaris
  \item Louis-Martin Rousseau %. Constraint Programming and OR. Lately working on using ML for OR/CP problems and solvers.
\end{itemize}

\end{document}
