\documentclass[10pt]{article}

\newcommand{\commentout}[1]{}
\usepackage[colorlinks=true,urlcolor=blue,citecolor=blue,pdfstartview=FitH]{hyperref}


\begin{document}

\title{ICAPS 2020 Workshop on \\ Bridging the Gap Between AI Planning and Reinforcement Learning (PRL)\\ \vspace*{0.7cm} Workshop Proposal
}
\date{}

\author{}

\maketitle


We propose to start a new workshop series at ICAPS 2020. The aim of this
workshop is to bridge the gap between two communities that mostly unaware of
each other while dealing with essentially the same problem of sequential
decision making. 

\section*{Workshop Description}


%\newpage

\section*{Relevance to ICAPS 2020}
ICAPS 2020 has seen a significant increase in amount of submissions on
Reinforcement Learning, indicating an increased interest from the RL community
towards Planning. 


\section*{Workshop Format}

The workshop is planned to have a full 1-day format, but the precise format is
to be decided as a function of the contributions received. We strive to have two
types of presentations: long and short (30 and 15 minutes, including
discussion), 1-2 invited talks, as well as 1-2 dedicated discussion sessions
where the audience members are encouraged to participate.

We would like to reduce the reviewing load of the community by not having a
program committee. Acceptance decisions will be made directly by the organizers.
We expect to have around 10-20 submissions, which can easily be handled by us,
particularly considering that reviewing for workshops is generally light.
Between us, we have sufficient competence in the topic of the workshop to make
informed decisions, but should the need arise, we can call on additional
reviewing expertise.

\section*{Organizers}

\begin{itemize}



\item \href{https://resedit.watson.ibm.com/researcher/view.php?person=ibm-Michael.Katz1}{Michael Katz}
 (\href{mailto:michael.katz1@ibm.com}{michael.katz1@ibm.com})\\
 Michael is a researcher at IBM T.J. Watson Research Center, NY, USA. His
 primary research interest is in heuristic search for domain independent planning.
 He received his PhD for a dissertation on heuristics for domain independent
 planning from Technion in 2010.
 He was a co-organizer of the 2011, 2013, 2014, 2015, 2016, and 2018 H(S)DIP
 workshops.


\item \href{http://hectorpalacios.net/}{Hector Palacios}
  (\href{mailto:hectorpal@gmail.com}{hectorpal@gmail.com})\\

  


\end{itemize}

\section*{Audience}



\end{document}
